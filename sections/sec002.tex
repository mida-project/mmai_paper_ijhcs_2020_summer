\section{Background}
\label{sec:related_work}

Two areas motivate this research: a) {\it radiomics}; and b) Human-AI interaction. In the following sections, we address each one.

\subsection{Radiomics}\label{sec:Radiomics}

In modern healthcare, medical images are crucial to support decision making, both in terms of diagnosis, predictions, and treatment planning~\cite{https://doi.org/10.13140/rg.2.2.25718.65606}.
In many situations, the accurate lesion detection and segmentation constitutes a mandatory requirement for image analysis. 
Over the last decades, several methods have been proposed to automatically perform these two tasks~\cite{litjens2017survey}.

One way to perform the detection and segmentation is based on the use of deep neural networks, depending on the amount of data used for training.
In a supervised (or semi-supervised) setting the training data should be annotated.
However this requirement is hard to be accomplished, since the annotations process is usually costly~\cite{https://doi.org/10.13140/rg.2.2.14792.55049, https://doi.org/10.13140/rg.2.2.16086.88649, 10.1145/3399715.3399744}.
Thus, one has to resort to small datasets.
To ensure that the networks' performance is not hampered by the lack of training images, transfer learning~\cite{NEURIPS2019_eb1e7832} is typically used, in which the network is first trained with a large unrelated data set and is then fine-tuned for the target task~\cite{shin2016deep}.

Recently, Convolutional Neural Networks (CNN) were shown to be effective at diagnosing breast cancer~\cite{carneiro2017automated}.
However, this and many other approaches \cite{becker2017deep, khan2019novel, wang2016discrimination} are limited to the MG modality.
Few works have taken advantage of multi-modality to further enhance the performance of these methods \cite{murtaza2019deep}.
Specifically in recent years, deep learning (DL) approaches have contributed to the notoriety of {\it AI-Assisted} healthcare~\cite{topol2019high}.
These methods have been outperforming older approaches and setting  new State-Of-The-Art results, with some achieving or even surpassing human-level performances~\cite{esteva2017dermatologist,gale2017detecting}.

The success of DL methods is due to the ability of the CNN~\cite{greenspan2016guest} to extract meaningful features, obtained using large training datasets.
These networks learn to extract and analyze large numbers of image-based features ({\it e.g.}, through {\it radiomics}) for quantitative characterization and analysis of tumor phenotype~\cite{10.1007/978-3-319-59876-5_7}.
Several deep neural architectures have been proposed, {\it e.g.}, DenseNet~\cite{huang2017densely} for classification (diagnosis) problems, or U-Net~\cite{ronneberger2015u} for segmentation problems.

Despite the success of DL, and even though different studies have shown that {\it radiomics} ({\it i.e.}, {\it AI-Assisted} methods for medical image analysis) can reduce human error and improve outcomes~\cite{Cai:2019:HTC:3290605.3300234,delvaux2017effects,middleton2016clinical}, their adoption by the medical community has been slow.
One of the main reasons is the inability of these systems to provide relevant medical information or to capture the nuances of the human mind~\cite{khairat2018reasons, kohli2018cad, 10.1145/2858036.2858373}, making them untrustworthy and preventing their clinical acceptance.
In particular, DL-based methods have been frequently viewed as black box approaches~\cite{litjens2017survey}.
Therefore, HCIs play an important role in creating user-friendly interactive systems for {\it AI-Assisted} medical image analysis~\cite{https://doi.org/10.13140/rg.2.2.30479.43682, Calisto:2017:TTM:3132272.3134111, https://doi.org/10.13140/rg.2.2.33421.59360}.

\subsection{Human-AI Interaction}

Applications of Human-AI collaboration in complex domains are subject to  the following two issues:
(1) trust, transparency and accountability of the involved AI agent~\cite{amershi2019guidelines}; and
(2) user's ability to understand and predict agent behavior, {\em i.e.}, explainability and intelligibility~\cite{Cai:2019:EEE:3301275.3302289, gunning2017explainable, 10.1007/978-3-319-99740-7_1, miller2018explanation}.
Forming accurate mental models of the {\it AI-Assisted} is useful for:
(i) representing the clinician's belief about what the system can do, acquired via interviews and observations, instruction, or inference;
(ii) mapping between the observable features of our system and the functionality perceived by the user; and (iii) the prediction for anticipating the AI output in a given scenario.

Whilst eXplainable AI (XAI)~\cite{10.5555/3365202} deals with the implementation of transparency and traceability of statistical `black‐box' machine learning methods.
Particularly, deep learning approaches are in certain domains, a pressing need to go beyond XAI; for example, to reach a level of explainable medicine there is a crucial need for causability.
However, causability~\cite{https://doi.org/10.1002/widm.1312} is different from causality.
In the same way that usability encompasses measurements for the quality of use, causability encompasses measurements for the quality of explanations produced by XAI methods ({\it e.g.}, the heatmaps of Section~\ref{sec:methods}).
Specifically, causability is the property of a human ({\it i.e.}, human intelligence), whereas explainability is a property of a system ({\it i.e.}, artificial intelligence).

In the medical domain it is of supreme importance to enable a domain expert to understand~\cite{yang2019profiling}, `why' an algorithm came up with a certain result ({\it e.g.}, this is necessary due to raising legal issues).
With certain XAI methods, such as layer-wise relevance propagation, relevant parts of inputs to, and representations in, a neural network which caused a result, can be highlighted (with a heatmap).
However, this is only a first - important - but only first step to ensure that end users, {\it e.g.}, medical professionals (humans), assume responsibility for decision making with AI.
The backbone for this approach is interactive ML~\cite{holzinger2016interactive}, which adds the component of human expertise to AI/ML processes by enabling them to re-enact and retrace the results on demand, {\it e.g.}, let them check it for plausibility.
This requires new human-AI interfaces for XAI and in order to do so, one has to deal with the question of how to evaluate the quality of explanations given by a XAI system, for this we need measurements, {\it e.g.}, the recently developed System Causability Scale - to measure the quality of explanations~\cite{andreas2020measuring}.

In this context, expectation theories~\cite{Kocielnik:2019:YAI:3290605.3300641, leung2019health} are postulating that user satisfaction and acceptance of a system is directly related to the difference between initial expectations and the actual clinical experience.
Specifically, expecting more than the system can deliver will decrease user satisfaction and lead to the rejection of the system.
Hence, we created our proposed technique, based on the following contributions:
(1) providing users a new control feature on the introduction of AI methods among medical imaging diagnosis; and
(2) the impact of the radiologists {\it behaviour} and the impact in professional practice.
We did that to achieve more accurate expectations of the systems capabilities addressing potential gaps.