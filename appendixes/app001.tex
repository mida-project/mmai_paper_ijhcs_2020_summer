% #############################################################################
% This is Appendix A
% !TEX root = main.tex
% #############################################################################

\section{Appendix}
\label{app:app001}

\hfill

\subsection{DenseNet Steps}
\label{app:steps}

\hfill

\noindent
The DenseNet steps are as follows:

\hfill

\begin{enumerate}
\item The DenseNet is pre-trained with the ImageNet model that contains roughly 1.2 millions images. This gives generalization capabilities to new datasets.
\item Next, we remove the last layer of the DenseNet (that contains a large number of classes, about 1000 classes or end-nodes) and replace it by a fully connected layer containing now three nodes. Each node of the network corresponds to one of the following classes: (i) low severity (Class 1: BIRADS <= 1 ), (ii) medium severity  (Class 2:  1 < BIRADS <= 3) and (iii) high severity (Class 3: BIRADS > 3). Thus, we have a DenseNet with three nodes in the last layer.
\hfill
\item Now, the pre-trained DenseNet goes through a process of fine-tuning in our breast dataset. The dataset comprises 1125 training images ({\it i.e.}, MG in both CC and MLO views, US, and MRI). Notice that here, we have 338 cases or patients (Section~\ref{sec:procedure}). The larger number of training images, when compared to the number of patients, stems from the fact that one patient may have more than one image.
\hfill
\item We split the above dataset in the following way: 80\% for training and the remaining 20\% for testing. In this partition, we guarantee that each set (training and test) are properly balanced, that is, each set contains samples that belong to all classes (recall that there are three classes as mentioned in Section~\ref{sec:procedure}, see also point 1) and all modalities ({\it i.e.}, MG, US, and MRI). Therefore, the experiments take into consideration the multimodality.
\hfill
\item The pre-training follows a supervised learning strategy. Specifically, during the training we provide the image, as well as the corresponding label, that is, the classification in one of the three classes for the image. This classification ({\it i.e.}, ground-truth) is provided by a set of eight radiologists (Section~\ref{sec:methods}). Say that this training stage gives to the DenseNet the ability to establish associations between the morphology of the lesion (image) and its severity (classification label).
\hfill
\item Finally, we use the held-out test set to measure the performance of the DenseNet. An accuracy of 98.2\% was obtained in this test set.
\hfill
\item As a final note, we should mention that the three patients (as mentioned in Section~\ref{sec:participants}) are obtained from the test set as mentioned above (see point 4). 
From the above the number of test samples is high, however, only three samples are taken  for the experiments purpose. Thus, the training and testing phases take into account a large number of samples.
\end{enumerate}

%%%%%%%%%%%%%%%%%%%%%%%%%%%%%%%%%%%%%%%%%%%%%%%%%%%
%\begin{table}[htbp]
\centering
\begin{tabular}{|c|c|c|c|c|}
\hline
\multirow{2}{*}{BIRADS} & \multicolumn{2}{c|}{Precision} & \multicolumn{2}{c|}{Recall}   \\ \cline{2-5} 
                        & Curre.          & Assis.       & Curre.         & Assis.    \\ \hline
1                       & 0.01            & 0.69         & 0.76           & 0.91         \\ \hline
2                       & 0.13            & 0.86         & 0.91           & 0.91         \\ \hline
3                       & 0.20            & 0.50         & 0.50           & 0.20         \\ \hline
4                       & 0.20            & 0.99         & 0.61           & 0.68         \\ \hline
5                       & 0.99            & 0.99         & 0.57           & 0.86         \\ \hline
\end{tabular}
\caption{Reported performance of {\it Current} (Curre.) and {\it Assistant} (Assis.) conditions. The {\it Precision} is the fraction of relevance instances among the classified instances. On the other hand, the {\it Recall} (sensitivity) is the fraction of relevant instances classified over the total amount of relevant instances.}
\label{tab:tab004}
\end{table}
%%%%%%%%%%%%%%%%%%%%%%%%%%%%%%%%%%%%%%%%%%%%%%%%%%%

%%%%%%%%%%%%%%%%%%%%%%%%%%%%%%%%%%%%%%%%%%%%%%%%%%%
\begin{landscape}
\begin{table}[htbp]
\resizebox{\textheight}{!}{%
\begin{tabular}{|r|c|l|l|l|l|l|}
\hline
\textbf{ID} & \textbf{Group} & \textbf{Speciality}              & \textbf{Medical Experience}                           & \textbf{Education}                            & \textbf{Work Sector} & \textbf{Institution}      \\ \hline
1           & Senior         & Head of Radiology Director       & more than 10 years of speciality                      & Medicinae Doctor (M.D.)                       & Public               & Hospital Fernando Fonseca \\ \hline
2           & Intern         & Medical General Internship       & doing medical internship                              & Bologna Master Degree                         & Public, Private      & Hospital Fernando Fonseca \\ \hline
3           & Intern         & Radiology Internship             & did a medical internship as Gynecologist and Oncology & Post-Bologna Degree and Bologna Master Degree & Public, Social       & Hospital Fernando Fonseca \\ \hline
4           & Intern         & Medical General Internship       & doing medical internship                              & Medicinae Doctor (M.D.)                       & Public               & Hospital Fernando Fonseca \\ \hline
5           & Intern         & Medical General Internship       & doing medical internship                              & Bologna Master Degree                         & Public               & Hospital Fernando Fonseca \\ \hline
6           & Intern         & Medical General Internship       & doing medical internship                              & Bologna Master Degree                         & Public               & Hospital Fernando Fonseca \\ \hline
7           & Junior         & Radiologist                      & a specialist with less than 5 years of speciality     & Medicinae Doctor (M.D.)                       & Public               & Hospital Fernando Fonseca \\ \hline
8           & Senior         & Radiologist                      & more than 10 years of speciality                      & Bologna Doctoral Degree (PhD)                 & Public, Private      & Hospital Fernando Fonseca \\ \hline
9           & Junior         & Radiologist                      & a specialist with less than 5 years of speciality     & Medicinae Doctor (M.D.)                       & Public               & Hospital Fernando Fonseca \\ \hline
10          & Senior         & Surgeon                          & more than 10 years of speciality                      & Medicinae Doctor (M.D.)                       & Public               & Hospital de Santa Maria   \\ \hline
11          & Middle         & Radiologist, Senology, Mastology & between 5 to 10 years of speciality                   & Pre-Bologna Degree and Specialist             & Private              & SAMS Hospital             \\ \hline
12          & Middle         & Immunotherapist                  & between 5 to 10 years of speciality                   & Medicinae Doctor (M.D.)                       & Private              & Madeira Medical Center    \\ \hline
13          & Middle         & Head of Radiology Director       & between 5 to 10 years of speciality                   & Medicinae Doctor (M.D.)                       & Public               & IPO Coimbra               \\ \hline
14          & Middle         & Radiology Coordinator            & between 5 to 10 years of speciality                   & Medicinae Doctor (M.D.)                       & Public               & IPO Lisboa                \\ \hline
15          & Intern         & Radiology Internship             & doing medical internship                              & Bologna Master Degree                         & Public               & Hospital do Barreiro      \\ \hline
16          & Intern         & Radiology Internship             & doing medical internship                              & Medicinae Doctor (M.D.)                       & Public               & IPO Lisboa                \\ \hline
17          & Middle         & Radiologist                      & between 5 to 10 years of speciality                   & Medicinae Doctor (M.D.)                       & Public               & IPO Lisboa                \\ \hline
18          & Intern         & Medical General Internship       & doing medical internship                              & Medicinae Doctor (M.D.)                       & Public               & IPO Lisboa                \\ \hline
19          & Middle         & Radiologist                      & between 5 to 10 years of speciality                   & Medicinae Doctor (M.D.)                       & Public               & IPO Lisboa                \\ \hline
20          & Senior         & Radiologist                      & more than 10 years of speciality                      & Medicinae Doctor (M.D.)                       & Public               & IPO Lisboa                \\ \hline
21          & Senior         & Radiologist                      & more than 10 years of speciality                      & Medicinae Doctor (M.D.)                       & Public               & IPO Coimbra               \\ \hline
22          & Middle         & Radiologist                      & between 5 to 10 years of speciality                   & Medicinae Doctor (M.D.)                       & Public               & IPO Coimbra               \\ \hline
23          & Middle         & Radiologist                      & between 5 to 10 years of speciality                   & Medicinae Doctor (M.D.)                       & Public               & IPO Coimbra               \\ \hline
24          & Middle         & Radiologist                      & between 5 to 10 years of speciality                   & Medicinae Doctor (M.D.)                       & Public               & IPO Coimbra               \\ \hline
25          & Middle         & Radiologist                      & between 5 to 10 years of speciality                   & Medicinae Doctor (M.D.)                       & Public               & IPO Coimbra               \\ \hline
26          & Middle         & Radiologist                      & between 5 to 10 years of speciality                   & Medicinae Doctor (M.D.)                       & Public               & IPO Coimbra               \\ \hline
27          & Senior         & Radiologist                      & more than 10 years of speciality                      & Medicinae Doctor (M.D.)                       & Public               & IPO Coimbra               \\ \hline
28          & Junior         & Radiologist                      & a specialist with less than 5 years of speciality     & Medicinae Doctor (M.D.)                       & Public               & Hospital Fernando Fonseca \\ \hline
29          & Senior         & Radiologist                      & more than 10 years of speciality                      & Medicinae Doctor (M.D.)                       & Public               & Hospital Fernando Fonseca \\ \hline
30          & Middle         & Radiologist                      & between 5 to 10 years of speciality                   & Medicinae Doctor (M.D.)                       & Public               & Hospital Fernando Fonseca \\ \hline
31          & Intern         & Radiology Internship             & doing medical internship                              & Medicinae Doctor (M.D.)                       & Public               & IPO Lisboa                \\ \hline
32          & Junior         & Radiologist                      & a specialist with less than 5 years of speciality     & Medicinae Doctor (M.D.)                       & Public               & Hospital Fernando Fonseca \\ \hline
33          & Intern         & Radiology Internship             & doing medical internship                              & Bologna Master Degree                         & Public               & Hospital do Barreiro      \\ \hline
34          & Middle         & Radiologist                      & between 5 to 10 years of speciality                   & Medicinae Doctor (M.D.)                       & Public               & Hospital do Barreiro      \\ \hline
35          & Junior         & Radiologist                      & a specialist with less than 5 years of speciality     & Medicinae Doctor (M.D.)                       & Public               & Hospital do Barreiro      \\ \hline
36          & Junior         & Radiologist                      & a specialist with less than 5 years of speciality     & Medicinae Doctor (M.D.)                       & Public               & Hospital do Barreiro      \\ \hline
37          & Intern         & Radiology Internship             & doing medical internship                              & Bologna Master Degree                         & Public, Private      & Hospital do Barreiro      \\ \hline
38          & Intern         & Radiology Internship             & doing medical internship                              & Medicinae Doctor (M.D.)                       & Public               & Hospital do Barreiro      \\ \hline
39          & Senior         & Radiologist                      & more than 10 years of speciality                      & Medicinae Doctor (M.D.)                       & Public               & Hospital do Barreiro      \\ \hline
40          & Senior         & Radiologist                      & more than 10 years of speciality                      & Medicinae Doctor (M.D.)                       & Public               & Hospital do Barreiro      \\ \hline
41          & Senior         & Radiologist                      & more than 10 years of speciality                      & Medicinae Doctor (M.D.)                       & Public               & IPO Lisboa                \\ \hline
42          & Junior         & Oncologist                       & a specialist with less than 5 years of speciality     & Medicinae Doctor (M.D.)                       & Public               & Hospital de Santa Maria   \\ \hline
43          & Middle         & Radiologist                      & between 5 to 10 years of speciality                   & Medicinae Doctor (M.D.)                       & Public               & IPO Lisboa                \\ \hline
44          & Junior         & Radiologist                      & a specialist with less than 5 years of speciality     & Bologna Master Degree                         & Public               & IPO Lisboa                \\ \hline
45          & Senior         & Radiologist                      & more than 10 years of speciality                      & Medicinae Doctor (M.D.)                       & Public               & IPO Coimbra               \\ \hline
\end{tabular}%
}
\caption{This table represents the demographic data of our participants. The {\bf ID} is the participants identifier for each clinician, {\it e.g.}, Clinician 1 (C\underline{1}), Clinicia 2 (C\underline{2}), ..., Clinician 45 (C\underline{45}). The {\bf Group} represents the professional medical experience of clinicians ({\it i.e.}, Intern, Junior, Middle or Senior) and the {\bf Speciality} represents the medical stage of speciality. The {\bf Medical Experience} represents the number of years working as a clinician or as an internship of medical studies. The {\bf Education} is the clinicians' level of education or background. The {\bf Work Sector} represents what type of place and the {\bf Institution} is the respective working host institution, as well as where the user tests where led for this study. (\href{https://mimbcd-ui.github.io/dataset-uta7-demographics}{mimbcd-ui.github.io/dataset-uta7-demographics})}
\label{tab:tab003}
\end{table}
\end{landscape}
%%%%%%%%%%%%%%%%%%%%%%%%%%%%%%%%%%%%%%%%%%%%%%%%%%%