\begin{table}[htbp]
\begin{tabular}{ c c c c c c c }
\toprule
\small
&
\multicolumn{2}{ c }{Eff.}
&
\multicolumn{2}{ c }{Per.}
&
\multicolumn{2}{ c }{Fru.}
\\
\cmidrule(lr){2-3}
\cmidrule(lr){4-5}
\cmidrule(lr){6-7}
Mod. & F & Sig. & F & Sig. & F & Sig. \\
\bottomrule
Curre. & 0.534 & 0.661 & 5.556 & 0.003$\star$ & 2.392 & 0.082 \\
Assis. & 0.664 & 0.578 & 0.319 & 0.811 & 0.408 & 0.748 \\
\bottomrule
\hfill
\end{tabular}
\caption{ The Analysis of Variance (ANOVA) factorial analysis table regarding NASA-TLX for \textit{Effort (Eff.)}, \textit{Performance (Per.)} and \textit{Frustration (Fru.)}, where {\it F} is the variation between sample means and the variation within samples. To determine whether any of the differences between the means are statistically significant, we used {\it Sig.} for significance. On the present study, we used a 20-point Likert Scale regarding Workload. The factorial analysis was described assuming $\alpha = 0.05$. Also, each time $p < 0.05$ it is marked with the $\star$ symbol.}
\label{tab:nasatlx_metrics}
\end{table}